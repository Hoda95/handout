\documentclass[
  man,
  floatsintext,
  longtable,
  nolmodern,
  notxfonts,
  notimes,
  colorlinks=true,linkcolor=blue,citecolor=blue,urlcolor=blue]{apa7}

\usepackage{amsmath}
\usepackage{amssymb}



\usepackage[bidi=default]{babel}
\babelprovide[main,import]{american}


% get rid of language-specific shorthands (see #6817):
\let\LanguageShortHands\languageshorthands
\def\languageshorthands#1{}

\RequirePackage{longtable}
\RequirePackage{threeparttablex}

\makeatletter
\renewcommand{\paragraph}{\@startsection{paragraph}{4}{\parindent}%
	{0\baselineskip \@plus 0.2ex \@minus 0.2ex}%
	{-.5em}%
	{\normalfont\normalsize\bfseries\typesectitle}}

\renewcommand{\subparagraph}[1]{\@startsection{subparagraph}{5}{0.5em}%
	{0\baselineskip \@plus 0.2ex \@minus 0.2ex}%
	{-\z@\relax}%
	{\normalfont\normalsize\bfseries\itshape\hspace{\parindent}{#1}\textit{\addperi}}{\relax}}
\makeatother




\usepackage{longtable, booktabs, multirow, multicol, colortbl, hhline, caption, array, float, xpatch}
\usepackage{subcaption}


\renewcommand\thesubfigure{\Alph{subfigure}}
\setcounter{topnumber}{2}
\setcounter{bottomnumber}{2}
\setcounter{totalnumber}{4}
\renewcommand{\topfraction}{0.85}
\renewcommand{\bottomfraction}{0.85}
\renewcommand{\textfraction}{0.15}
\renewcommand{\floatpagefraction}{0.7}

\usepackage{tcolorbox}
\tcbuselibrary{listings,theorems, breakable, skins}
\usepackage{fontawesome5}

\definecolor{quarto-callout-color}{HTML}{909090}
\definecolor{quarto-callout-note-color}{HTML}{0758E5}
\definecolor{quarto-callout-important-color}{HTML}{CC1914}
\definecolor{quarto-callout-warning-color}{HTML}{EB9113}
\definecolor{quarto-callout-tip-color}{HTML}{00A047}
\definecolor{quarto-callout-caution-color}{HTML}{FC5300}
\definecolor{quarto-callout-color-frame}{HTML}{ACACAC}
\definecolor{quarto-callout-note-color-frame}{HTML}{4582EC}
\definecolor{quarto-callout-important-color-frame}{HTML}{D9534F}
\definecolor{quarto-callout-warning-color-frame}{HTML}{F0AD4E}
\definecolor{quarto-callout-tip-color-frame}{HTML}{02B875}
\definecolor{quarto-callout-caution-color-frame}{HTML}{FD7E14}

%\newlength\Oldarrayrulewidth
%\newlength\Oldtabcolsep


\usepackage{hyperref}




\providecommand{\tightlist}{%
  \setlength{\itemsep}{0pt}\setlength{\parskip}{0pt}}
\usepackage{longtable,booktabs,array}
\usepackage{calc} % for calculating minipage widths
% Correct order of tables after \paragraph or \subparagraph
\usepackage{etoolbox}
\makeatletter
\patchcmd\longtable{\par}{\if@noskipsec\mbox{}\fi\par}{}{}
\makeatother
% Allow footnotes in longtable head/foot
\IfFileExists{footnotehyper.sty}{\usepackage{footnotehyper}}{\usepackage{footnote}}
\makesavenoteenv{longtable}

\usepackage{graphicx}
\makeatletter
\newsavebox\pandoc@box
\newcommand*\pandocbounded[1]{% scales image to fit in text height/width
  \sbox\pandoc@box{#1}%
  \Gscale@div\@tempa{\textheight}{\dimexpr\ht\pandoc@box+\dp\pandoc@box\relax}%
  \Gscale@div\@tempb{\linewidth}{\wd\pandoc@box}%
  \ifdim\@tempb\p@<\@tempa\p@\let\@tempa\@tempb\fi% select the smaller of both
  \ifdim\@tempa\p@<\p@\scalebox{\@tempa}{\usebox\pandoc@box}%
  \else\usebox{\pandoc@box}%
  \fi%
}
% Set default figure placement to htbp
\def\fps@figure{htbp}
\makeatother







\usepackage{newtx}

\defaultfontfeatures{Scale=MatchLowercase}
\defaultfontfeatures[\rmfamily]{Ligatures=TeX,Scale=1}





\title{R Control Flow Statements: Building blocks for automated Decision
Making}


\shorttitle{R Control}


\usepackage{etoolbox}






\author{Hoda Haeri (Matriculation:400963829)}



\affiliation{
{Hochschule Fresenius - University of Applied Science}}




\leftheader{(Matriculation:400963829)}



\abstract{R is a powerful programming language widely used for data
analysis and visualization. Control flow statements in R---such as if,
else, for, while, and repeat---allow users to automate decision-making
and repetitive tasks. These statements are the core building blocks that
enable scripts to respond to data, adapt to changing situations, and
streamline complex analytical processes. This document provides a clear
overview of the main control flow statements in R, their syntax, and
practical examples to illustrate their role in automating
decision-making. }

\keywords{R programming, Control flow statements, Automation, Data
analysis, Decision-making}

\authornote{ 
\par{ }
\par{   The authors have no conflicts of interest to disclose.    }
\par{Correspondence concerning this article should be addressed to Hoda
Haeri
(Matriculation:400963829), Email: \href{mailto:haeri.hoda@stud.hs-fresenius.de}{haeri.hoda@stud.hs-fresenius.de}}
}

\makeatletter
\let\endoldlt\endlongtable
\def\endlongtable{
\hline
\endoldlt
}
\makeatother

\urlstyle{same}



\makeatletter
\@ifpackageloaded{caption}{}{\usepackage{caption}}
\AtBeginDocument{%
\ifdefined\contentsname
  \renewcommand*\contentsname{Table of contents}
\else
  \newcommand\contentsname{Table of contents}
\fi
\ifdefined\listfigurename
  \renewcommand*\listfigurename{List of Figures}
\else
  \newcommand\listfigurename{List of Figures}
\fi
\ifdefined\listtablename
  \renewcommand*\listtablename{List of Tables}
\else
  \newcommand\listtablename{List of Tables}
\fi
\ifdefined\figurename
  \renewcommand*\figurename{Figure}
\else
  \newcommand\figurename{Figure}
\fi
\ifdefined\tablename
  \renewcommand*\tablename{Table}
\else
  \newcommand\tablename{Table}
\fi
}
\@ifpackageloaded{float}{}{\usepackage{float}}
\floatstyle{ruled}
\@ifundefined{c@chapter}{\newfloat{codelisting}{h}{lop}}{\newfloat{codelisting}{h}{lop}[chapter]}
\floatname{codelisting}{Listing}
\newcommand*\listoflistings{\listof{codelisting}{List of Listings}}
\makeatother
\makeatletter
\makeatother
\makeatletter
\@ifpackageloaded{caption}{}{\usepackage{caption}}
\@ifpackageloaded{subcaption}{}{\usepackage{subcaption}}
\makeatother

% From https://tex.stackexchange.com/a/645996/211326
%%% apa7 doesn't want to add appendix section titles in the toc
%%% let's make it do it
\makeatletter
\xpatchcmd{\appendix}
  {\par}
  {\addcontentsline{toc}{section}{\@currentlabelname}\par}
  {}{}
\makeatother

%% Disable longtable counter
%% https://tex.stackexchange.com/a/248395/211326

\usepackage{etoolbox}

\makeatletter
\patchcmd{\LT@caption}
  {\bgroup}
  {\bgroup\global\LTpatch@captiontrue}
  {}{}
\patchcmd{\longtable}
  {\par}
  {\par\global\LTpatch@captionfalse}
  {}{}
\apptocmd{\endlongtable}
  {\ifLTpatch@caption\else\addtocounter{table}{-1}\fi}
  {}{}
\newif\ifLTpatch@caption
\makeatother

\begin{document}

\maketitle

\hypertarget{toc}{}
\tableofcontents
\newpage
\section[Introduction]{R Control Flow Statements: Building blocks for
automated Decision Making}

\setcounter{secnumdepth}{-\maxdimen} % remove section numbering

\setlength\LTleft{0pt}


\textbf{R} is a widely used language for statistical analysis, data
manipulation, and visualization. One of its most powerful features is
the ability to automate decisions and repetitive tasks through control
flow statements, such as if, else, for, while, and repeat. These
statements act like logic gates in a program, directing how and when
different parts of the code are executed, based on specific conditions
or data values. This approach allows R scripts to efficiently manage
various types of data, respond to unexpected situations like missing
values, and repeat processes without manual effort, making analysis more
robust and scalable

\textbf{What Are Control Flow Statements in R?}

Control flow statements in R work much like traffic signals, determining
which sections of code should run, depending on the data or situation.
By using these statements, programs can automatically make decisions,
choose different paths of analysis, and repeat tasks as needed. This
makes R code not only dynamic and efficient but also reliable, even when
faced with real-world data challenges

\textbf{Types of Control Flow Statements}

1- If, Else If, and Else Statements:

\textbf{If} statements execute a block of code only if a specified
condition is true.

\textbf{Else} statements execute \textbf{if} the if condition is false.

\textbf{Else if} statements check further conditions if the previous
ones are not met.

2- Repeat Loops:

Repeat loops execute a block of code multiple times until a certain
condition is met, using a break statement to stop the loop.

3- While Loops:

While loops continue to run as long as a specific condition remains
true, making them useful for processing unknown or varying amounts of
data.

4- For Loops:

For loops repeat a block of code a set number of times, usually once for
each value in a vector or list, making them ideal for systematic,
repeated tasks.

\textbf{Why Are Control Flow Statements Important?}

Control flow statements are essential for making R scripts flexible,
automated, and intelligent. They enable automation in common tasks such
as data cleaning, handling missing values, and applying decision rules.
This means R can process and analyze data on its own, reducing the need
for manual oversight

\textbf{Conclusion}

Control flow statements are the building blocks that transform simple R
scripts into dynamic, responsive tools. By using statements like if,
else, for, while, and repeat, R users can automate decisions and
repetitive processes, making their data analysis workflows more powerful
and efficient. This adaptability is crucial for real-world data
analysis, where data and requirements often change

\section{Affidavit}\label{affidavit}

I hereby affirm that this submitted paper was authored unaided and
solely by me. Additionally, no other sources than those in the reference
list were used. Parts of this paper, including tables and figures, that
have been taken either verbatim or analogously from other works have in
each case been properly cited with regard to their origin and
authorship. This paper either in parts or in its entirety, be it in the
same or similar form, has not been submitted to any other examination
board and has not been published.

I acknowledge that the university may use plagiarism detection software
to check my thesis. I agree to cooperate with any investigation of
suspected plagiarism and to provide any additional information or
evidence requested by the university.

Checklist:

\begin{itemize}
\tightlist
\item[$\boxtimes$]
  The handout contains 3-5 pages of text.
\item[$\boxtimes$]
  The submission contains the Quarto file of the handout.
\item[$\boxtimes$]
  The submission contains the Quarto file of the presentation.
\item[$\boxtimes$]
  The submission contains the HTML file of the handout.
\item[$\boxtimes$]
  The submission contains the HTML file of the presentation.
\item[$\boxtimes$]
  The submission contains the PDF file of the handout.
\item[$\boxtimes$]
  The submission contains the PDF file of the presentation.
\item[$\boxtimes$]
  The title page of the presentation and the handout contain personal
  details (name, email, matriculation number).
\item[$\boxtimes$]
  The handout contains a abstract.
\item[$\boxtimes$]
  The presentation and the handout contain a bibliography, created using
  BibTeX with APA citation style.
\item[$\boxtimes$]
  Either the handout or the presentation contains R code that proof the
  expertise in coding.
\item[$\boxtimes$]
  The handout includes an introduction to guide the reader and a
  conclusion summarizing the work and discussing potential further
  investigations and readings, respectively.
\item[$\boxtimes$]
  All significant resources used in the report and R code development.
\item[$\boxtimes$]
  The filled out Affidavit.
\item[$\boxtimes$]
  A concise description of the successful use of Git and GitHub, as
  detailed here: \url{https://github.com/hubchev/make_a_pull_request}.
\item[$\boxtimes$]
  The link to the presentation and the handout published on GitHub.
\end{itemize}

{[}Hoda Haeri,{]} {[}06/18/2025,{]} {[}Köln{]}






\end{document}

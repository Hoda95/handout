\documentclass[
  man,
  floatsintext,
  longtable,
  nolmodern,
  notxfonts,
  notimes,
  colorlinks=true,linkcolor=blue,citecolor=blue,urlcolor=blue]{apa7}

\usepackage{amsmath}
\usepackage{amssymb}



\usepackage[bidi=default]{babel}
\babelprovide[main,import]{american}


% get rid of language-specific shorthands (see #6817):
\let\LanguageShortHands\languageshorthands
\def\languageshorthands#1{}

\RequirePackage{longtable}
\RequirePackage{threeparttablex}

\makeatletter
\renewcommand{\paragraph}{\@startsection{paragraph}{4}{\parindent}%
	{0\baselineskip \@plus 0.2ex \@minus 0.2ex}%
	{-.5em}%
	{\normalfont\normalsize\bfseries\typesectitle}}

\renewcommand{\subparagraph}[1]{\@startsection{subparagraph}{5}{0.5em}%
	{0\baselineskip \@plus 0.2ex \@minus 0.2ex}%
	{-\z@\relax}%
	{\normalfont\normalsize\bfseries\itshape\hspace{\parindent}{#1}\textit{\addperi}}{\relax}}
\makeatother




\usepackage{longtable, booktabs, multirow, multicol, colortbl, hhline, caption, array, float, xpatch}
\usepackage{subcaption}


\renewcommand\thesubfigure{\Alph{subfigure}}
\setcounter{topnumber}{2}
\setcounter{bottomnumber}{2}
\setcounter{totalnumber}{4}
\renewcommand{\topfraction}{0.85}
\renewcommand{\bottomfraction}{0.85}
\renewcommand{\textfraction}{0.15}
\renewcommand{\floatpagefraction}{0.7}

\usepackage{tcolorbox}
\tcbuselibrary{listings,theorems, breakable, skins}
\usepackage{fontawesome5}

\definecolor{quarto-callout-color}{HTML}{909090}
\definecolor{quarto-callout-note-color}{HTML}{0758E5}
\definecolor{quarto-callout-important-color}{HTML}{CC1914}
\definecolor{quarto-callout-warning-color}{HTML}{EB9113}
\definecolor{quarto-callout-tip-color}{HTML}{00A047}
\definecolor{quarto-callout-caution-color}{HTML}{FC5300}
\definecolor{quarto-callout-color-frame}{HTML}{ACACAC}
\definecolor{quarto-callout-note-color-frame}{HTML}{4582EC}
\definecolor{quarto-callout-important-color-frame}{HTML}{D9534F}
\definecolor{quarto-callout-warning-color-frame}{HTML}{F0AD4E}
\definecolor{quarto-callout-tip-color-frame}{HTML}{02B875}
\definecolor{quarto-callout-caution-color-frame}{HTML}{FD7E14}

%\newlength\Oldarrayrulewidth
%\newlength\Oldtabcolsep


\usepackage{hyperref}



\usepackage{color}
\usepackage{fancyvrb}
\newcommand{\VerbBar}{|}
\newcommand{\VERB}{\Verb[commandchars=\\\{\}]}
\DefineVerbatimEnvironment{Highlighting}{Verbatim}{commandchars=\\\{\}}
% Add ',fontsize=\small' for more characters per line
\usepackage{framed}
\definecolor{shadecolor}{RGB}{241,243,245}
\newenvironment{Shaded}{\begin{snugshade}}{\end{snugshade}}
\newcommand{\AlertTok}[1]{\textcolor[rgb]{0.68,0.00,0.00}{#1}}
\newcommand{\AnnotationTok}[1]{\textcolor[rgb]{0.37,0.37,0.37}{#1}}
\newcommand{\AttributeTok}[1]{\textcolor[rgb]{0.40,0.45,0.13}{#1}}
\newcommand{\BaseNTok}[1]{\textcolor[rgb]{0.68,0.00,0.00}{#1}}
\newcommand{\BuiltInTok}[1]{\textcolor[rgb]{0.00,0.23,0.31}{#1}}
\newcommand{\CharTok}[1]{\textcolor[rgb]{0.13,0.47,0.30}{#1}}
\newcommand{\CommentTok}[1]{\textcolor[rgb]{0.37,0.37,0.37}{#1}}
\newcommand{\CommentVarTok}[1]{\textcolor[rgb]{0.37,0.37,0.37}{\textit{#1}}}
\newcommand{\ConstantTok}[1]{\textcolor[rgb]{0.56,0.35,0.01}{#1}}
\newcommand{\ControlFlowTok}[1]{\textcolor[rgb]{0.00,0.23,0.31}{\textbf{#1}}}
\newcommand{\DataTypeTok}[1]{\textcolor[rgb]{0.68,0.00,0.00}{#1}}
\newcommand{\DecValTok}[1]{\textcolor[rgb]{0.68,0.00,0.00}{#1}}
\newcommand{\DocumentationTok}[1]{\textcolor[rgb]{0.37,0.37,0.37}{\textit{#1}}}
\newcommand{\ErrorTok}[1]{\textcolor[rgb]{0.68,0.00,0.00}{#1}}
\newcommand{\ExtensionTok}[1]{\textcolor[rgb]{0.00,0.23,0.31}{#1}}
\newcommand{\FloatTok}[1]{\textcolor[rgb]{0.68,0.00,0.00}{#1}}
\newcommand{\FunctionTok}[1]{\textcolor[rgb]{0.28,0.35,0.67}{#1}}
\newcommand{\ImportTok}[1]{\textcolor[rgb]{0.00,0.46,0.62}{#1}}
\newcommand{\InformationTok}[1]{\textcolor[rgb]{0.37,0.37,0.37}{#1}}
\newcommand{\KeywordTok}[1]{\textcolor[rgb]{0.00,0.23,0.31}{\textbf{#1}}}
\newcommand{\NormalTok}[1]{\textcolor[rgb]{0.00,0.23,0.31}{#1}}
\newcommand{\OperatorTok}[1]{\textcolor[rgb]{0.37,0.37,0.37}{#1}}
\newcommand{\OtherTok}[1]{\textcolor[rgb]{0.00,0.23,0.31}{#1}}
\newcommand{\PreprocessorTok}[1]{\textcolor[rgb]{0.68,0.00,0.00}{#1}}
\newcommand{\RegionMarkerTok}[1]{\textcolor[rgb]{0.00,0.23,0.31}{#1}}
\newcommand{\SpecialCharTok}[1]{\textcolor[rgb]{0.37,0.37,0.37}{#1}}
\newcommand{\SpecialStringTok}[1]{\textcolor[rgb]{0.13,0.47,0.30}{#1}}
\newcommand{\StringTok}[1]{\textcolor[rgb]{0.13,0.47,0.30}{#1}}
\newcommand{\VariableTok}[1]{\textcolor[rgb]{0.07,0.07,0.07}{#1}}
\newcommand{\VerbatimStringTok}[1]{\textcolor[rgb]{0.13,0.47,0.30}{#1}}
\newcommand{\WarningTok}[1]{\textcolor[rgb]{0.37,0.37,0.37}{\textit{#1}}}

\providecommand{\tightlist}{%
  \setlength{\itemsep}{0pt}\setlength{\parskip}{0pt}}
\usepackage{longtable,booktabs,array}
\usepackage{calc} % for calculating minipage widths
% Correct order of tables after \paragraph or \subparagraph
\usepackage{etoolbox}
\makeatletter
\patchcmd\longtable{\par}{\if@noskipsec\mbox{}\fi\par}{}{}
\makeatother
% Allow footnotes in longtable head/foot
\IfFileExists{footnotehyper.sty}{\usepackage{footnotehyper}}{\usepackage{footnote}}
\makesavenoteenv{longtable}

\usepackage{graphicx}
\makeatletter
\newsavebox\pandoc@box
\newcommand*\pandocbounded[1]{% scales image to fit in text height/width
  \sbox\pandoc@box{#1}%
  \Gscale@div\@tempa{\textheight}{\dimexpr\ht\pandoc@box+\dp\pandoc@box\relax}%
  \Gscale@div\@tempb{\linewidth}{\wd\pandoc@box}%
  \ifdim\@tempb\p@<\@tempa\p@\let\@tempa\@tempb\fi% select the smaller of both
  \ifdim\@tempa\p@<\p@\scalebox{\@tempa}{\usebox\pandoc@box}%
  \else\usebox{\pandoc@box}%
  \fi%
}
% Set default figure placement to htbp
\def\fps@figure{htbp}
\makeatother


% definitions for citeproc citations
\NewDocumentCommand\citeproctext{}{}
\NewDocumentCommand\citeproc{mm}{%
  \begingroup\def\citeproctext{#2}\cite{#1}\endgroup}
\makeatletter
 % allow citations to break across lines
 \let\@cite@ofmt\@firstofone
 % avoid brackets around text for \cite:
 \def\@biblabel#1{}
 \def\@cite#1#2{{#1\if@tempswa , #2\fi}}
\makeatother
\newlength{\cslhangindent}
\setlength{\cslhangindent}{1.5em}
\newlength{\csllabelwidth}
\setlength{\csllabelwidth}{3em}
\newenvironment{CSLReferences}[2] % #1 hanging-indent, #2 entry-spacing
 {\begin{list}{}{%
  \setlength{\itemindent}{0pt}
  \setlength{\leftmargin}{0pt}
  \setlength{\parsep}{0pt}
  % turn on hanging indent if param 1 is 1
  \ifodd #1
   \setlength{\leftmargin}{\cslhangindent}
   \setlength{\itemindent}{-1\cslhangindent}
  \fi
  % set entry spacing
  \setlength{\itemsep}{#2\baselineskip}}}
 {\end{list}}
\usepackage{calc}
\newcommand{\CSLBlock}[1]{\hfill\break\parbox[t]{\linewidth}{\strut\ignorespaces#1\strut}}
\newcommand{\CSLLeftMargin}[1]{\parbox[t]{\csllabelwidth}{\strut#1\strut}}
\newcommand{\CSLRightInline}[1]{\parbox[t]{\linewidth - \csllabelwidth}{\strut#1\strut}}
\newcommand{\CSLIndent}[1]{\hspace{\cslhangindent}#1}





\usepackage{newtx}

\defaultfontfeatures{Scale=MatchLowercase}
\defaultfontfeatures[\rmfamily]{Ligatures=TeX,Scale=1}





\title{R Control Flow Statements: Building blocks for automated Decision
Making}


\shorttitle{R Control}


\usepackage{etoolbox}






\author{Hoda Haeri (Matriculation:400963829)}



\affiliation{
{Hochschule Fresenius - University of Applied Science}}




\leftheader{(Matriculation:400963829)}



\abstract{R is a powerful programming language widely used for data
analysis and visualization. Control flow statements in R---such as if,
else, for, while, and repeat---allow users to automate decision-making
and repetitive tasks. These statements are the core building blocks that
enable scripts to respond to data, adapt to changing situations, and
streamline complex analytical processes. This document provides a clear
overview of the main control flow statements in R, their syntax, and
practical examples to illustrate their role in automating
decision-making. }

\keywords{R programming, Control flow statements, Automation, Data
analysis, Decision-making}

\authornote{ 
\par{ }
\par{   The authors have no conflicts of interest to disclose.    }
\par{Correspondence concerning this article should be addressed to Hoda
Haeri
(Matriculation:400963829), Email: \href{mailto:haeri.hoda@stud.hs-fresenius.de}{haeri.hoda@stud.hs-fresenius.de}}
}

\makeatletter
\let\endoldlt\endlongtable
\def\endlongtable{
\hline
\endoldlt
}
\makeatother

\urlstyle{same}



\makeatletter
\@ifpackageloaded{caption}{}{\usepackage{caption}}
\AtBeginDocument{%
\ifdefined\contentsname
  \renewcommand*\contentsname{Table of contents}
\else
  \newcommand\contentsname{Table of contents}
\fi
\ifdefined\listfigurename
  \renewcommand*\listfigurename{List of Figures}
\else
  \newcommand\listfigurename{List of Figures}
\fi
\ifdefined\listtablename
  \renewcommand*\listtablename{List of Tables}
\else
  \newcommand\listtablename{List of Tables}
\fi
\ifdefined\figurename
  \renewcommand*\figurename{Figure}
\else
  \newcommand\figurename{Figure}
\fi
\ifdefined\tablename
  \renewcommand*\tablename{Table}
\else
  \newcommand\tablename{Table}
\fi
}
\@ifpackageloaded{float}{}{\usepackage{float}}
\floatstyle{ruled}
\@ifundefined{c@chapter}{\newfloat{codelisting}{h}{lop}}{\newfloat{codelisting}{h}{lop}[chapter]}
\floatname{codelisting}{Listing}
\newcommand*\listoflistings{\listof{codelisting}{List of Listings}}
\makeatother
\makeatletter
\makeatother
\makeatletter
\@ifpackageloaded{caption}{}{\usepackage{caption}}
\@ifpackageloaded{subcaption}{}{\usepackage{subcaption}}
\makeatother

% From https://tex.stackexchange.com/a/645996/211326
%%% apa7 doesn't want to add appendix section titles in the toc
%%% let's make it do it
\makeatletter
\xpatchcmd{\appendix}
  {\par}
  {\addcontentsline{toc}{section}{\@currentlabelname}\par}
  {}{}
\makeatother

%% Disable longtable counter
%% https://tex.stackexchange.com/a/248395/211326

\usepackage{etoolbox}

\makeatletter
\patchcmd{\LT@caption}
  {\bgroup}
  {\bgroup\global\LTpatch@captiontrue}
  {}{}
\patchcmd{\longtable}
  {\par}
  {\par\global\LTpatch@captionfalse}
  {}{}
\apptocmd{\endlongtable}
  {\ifLTpatch@caption\else\addtocounter{table}{-1}\fi}
  {}{}
\newif\ifLTpatch@caption
\makeatother

\begin{document}

\maketitle

\hypertarget{toc}{}
\tableofcontents
\newpage
\section[Introduction]{R Control Flow Statements: Building blocks for
automated Decision Making}

\setcounter{secnumdepth}{-\maxdimen} % remove section numbering

\setlength\LTleft{0pt}


\textbf{R} is a popular language for statistical analysis, data
manipulation, and visualization. One of its key strengths lies in its
ability to automate tasks and decision-making processes through the use
of control flow statements, such as if, else, for, while, and repeat.
These control flow statements act as the logic gates of a program,
directing the execution path based on specific conditions or data
values. This allows R scripts to handle various types of data, respond
to unexpected situations (like missing or unusual values), and
efficiently repeat processes without manual intervention. By
incorporating control flow statements, programmers can make their code
more dynamic, efficient, and capable of adapting to different analytical
requirements and datasets . This flexibility is essential for robust
data analysis and building scalable data processing pipelines in R.

(\citeproc{ref-kabacoff2022r}{Kabacoff, 2022})

\textbf{What Are Control Flow Statements in R?}

Control flow statements in R act much like traffic signals in a
city---they control which parts of your code are allowed to execute and
when, based on the current state of your data or specific conditions. By
using these statements, R programs can make decisions automatically,
such as choosing different analysis paths depending on data values. They
also allow for repeated actions, like processing every item in a
dataset, without requiring manual input each time. Additionally, control
flow statements make your code more robust by providing ways to handle
unexpected situations, such as missing or unusual data, ensuring that
your scripts remain accurate and reliable even when faced with
real-world data challenges (\citeproc{ref-kabacoff2022r}{Kabacoff,
2022}).

\textbf{Types of Control Flow Statements}

1- If, Else If, and Else Statements

\begin{itemize}
\tightlist
\item
  If statements execute a block of code only if a specified condition is
  true.
\end{itemize}

\textbf{Example:}

\begin{Shaded}
\begin{Highlighting}[]
\NormalTok{x }\OtherTok{\textless{}{-}} \DecValTok{10}

\ControlFlowTok{if}\NormalTok{ (x }\SpecialCharTok{\textgreater{}} \DecValTok{5}\NormalTok{) \{}
  \FunctionTok{print}\NormalTok{(}\StringTok{"x is greater than 5"}\NormalTok{)}
\NormalTok{\}}
\end{Highlighting}
\end{Shaded}

\begin{verbatim}
[1] "x is greater than 5"
\end{verbatim}

This code will print ``x is greater than 5'' because the condition is
true.

\begin{itemize}
\tightlist
\item
  Else statements execute when the condition in the if statement is
  false.
\end{itemize}

\textbf{Example:}

\begin{Shaded}
\begin{Highlighting}[]
\NormalTok{x }\OtherTok{\textless{}{-}} \DecValTok{3}

\ControlFlowTok{if}\NormalTok{ (x }\SpecialCharTok{\textgreater{}} \DecValTok{5}\NormalTok{) \{}
  \FunctionTok{print}\NormalTok{(}\StringTok{"x is greater than 5"}\NormalTok{)}
\NormalTok{\} }\ControlFlowTok{else}\NormalTok{ \{}
  \FunctionTok{print}\NormalTok{(}\StringTok{"x is not greater than 5"}\NormalTok{)}
\NormalTok{\}}
\end{Highlighting}
\end{Shaded}

\begin{verbatim}
[1] "x is not greater than 5"
\end{verbatim}

This code will print ``x is not greater than 5'' because the condition
is false and the else block runs.

\begin{itemize}
\tightlist
\item
  Else If statements check additional conditions if the previous
  conditions were false.
\end{itemize}

\textbf{Example:}

\begin{Shaded}
\begin{Highlighting}[]
\NormalTok{x }\OtherTok{\textless{}{-}} \DecValTok{5}

\ControlFlowTok{if}\NormalTok{ (x }\SpecialCharTok{\textgreater{}} \DecValTok{5}\NormalTok{) \{}
  \FunctionTok{print}\NormalTok{(}\StringTok{"x is greater than 5"}\NormalTok{)}
\NormalTok{\} }\ControlFlowTok{else} \ControlFlowTok{if}\NormalTok{ (x }\SpecialCharTok{==} \DecValTok{5}\NormalTok{) \{}
  \FunctionTok{print}\NormalTok{(}\StringTok{"x is exactly 5"}\NormalTok{)}
\NormalTok{\} }\ControlFlowTok{else}\NormalTok{ \{}
  \FunctionTok{print}\NormalTok{(}\StringTok{"x is less than 5"}\NormalTok{)}
\NormalTok{\}}
\end{Highlighting}
\end{Shaded}

\begin{verbatim}
[1] "x is exactly 5"
\end{verbatim}

This code will print ``x is exactly 5'' because the first condition is
false but the else if condition is true.

2- Repeat Loops

Repeat loops execute a block of code until a specific condition is met,
usually ending with a break statement to prevent infinite looping.

(\citeproc{ref-kabacoff2022r}{Kabacoff, 2022})

\textbf{Example:}

\begin{Shaded}
\begin{Highlighting}[]
\NormalTok{count }\OtherTok{\textless{}{-}} \DecValTok{1}

\ControlFlowTok{repeat}\NormalTok{ \{}
  \FunctionTok{print}\NormalTok{(count)}
\NormalTok{  count }\OtherTok{\textless{}{-}}\NormalTok{ count }\SpecialCharTok{+} \DecValTok{1}
  \ControlFlowTok{if}\NormalTok{ (count }\SpecialCharTok{\textgreater{}} \DecValTok{5}\NormalTok{) \{}
    \ControlFlowTok{break}
\NormalTok{  \}}
\NormalTok{\}}
\end{Highlighting}
\end{Shaded}

\begin{verbatim}
[1] 1
[1] 2
[1] 3
[1] 4
[1] 5
\end{verbatim}

This code prints numbers 1 to 5. The repeat loop keeps running until the
condition inside the loop (count \textgreater{} 5) becomes true, which
triggers the break statement and stops the loop.

(\citeproc{ref-grolemund2014hands}{Grolemund, 2014})

3- While Loops

While loops repeat a block of code as long as a certain condition
remains true.

\textbf{Example:}

\begin{Shaded}
\begin{Highlighting}[]
\NormalTok{count }\OtherTok{\textless{}{-}} \DecValTok{1}

\ControlFlowTok{while}\NormalTok{ (count }\SpecialCharTok{\textless{}=} \DecValTok{5}\NormalTok{) \{}
  \FunctionTok{print}\NormalTok{(count)}
\NormalTok{  count }\OtherTok{\textless{}{-}}\NormalTok{ count }\SpecialCharTok{+} \DecValTok{1}
\NormalTok{\}}
\end{Highlighting}
\end{Shaded}

\begin{verbatim}
[1] 1
[1] 2
[1] 3
[1] 4
[1] 5
\end{verbatim}

This code also prints numbers 1 to 5. The loop runs as long as the
condition (count \textless= 5) is true. Once count becomes greater than
5, the loop stops.

(\citeproc{ref-grolemund2014hands}{Grolemund, 2014})

4- For Loops

For loops execute a block of code a specific number of times, usually
once for each value in a vector.

\textbf{Example:}

\begin{Shaded}
\begin{Highlighting}[]
\NormalTok{numbers }\OtherTok{\textless{}{-}} \FunctionTok{c}\NormalTok{(}\DecValTok{1}\NormalTok{, }\DecValTok{2}\NormalTok{, }\DecValTok{3}\NormalTok{, }\DecValTok{4}\NormalTok{, }\DecValTok{5}\NormalTok{)}

\ControlFlowTok{for}\NormalTok{ (num }\ControlFlowTok{in}\NormalTok{ numbers) \{}
  \FunctionTok{print}\NormalTok{(num)}
\NormalTok{\}}
\end{Highlighting}
\end{Shaded}

\begin{verbatim}
[1] 1
[1] 2
[1] 3
[1] 4
[1] 5
\end{verbatim}

This code prints each number in the vector numbers, one by one. The loop
repeats for every value in the vector.

(\citeproc{ref-r_core_team2024}{R Core Team, 2024})

\subsection{Why Control Flow Statements Are Important for Automated
Decision-Making}\label{why-control-flow-statements-are-important-for-automated-decision-making}

Control flow statements make R scripts ``think'' for themselves,
adapting to new data and situations. They allow for automation in data
cleaning, handling missing values, and creating decision rules . For
example, you can write code that automatically fills in missing values
or flags unusual entries---tasks that would otherwise require manual
review.

(\citeproc{ref-kabacoff2022r}{Kabacoff, 2022})

\section{Conclusion}\label{conclusion}

Control flow statements are fundamental for anyone using R because they
turn simple, linear scripts into adaptable and interactive programs.
With control flow tools like if, else, for, while, and repeat, R users
can build scripts that make their own decisions and handle repetitive
tasks automatically. This capability allows the automation of complex
data analysis steps---such as checking for outliers, filling in missing
data, or running calculations across entire datasets---without manual
intervention. As a result, workflows become not only more efficient but
also more powerful, since the code can adjust to changing data and
requirements on its own . This adaptability is especially valuable in
real-world data analysis, where conditions and data structures often
change.

(\citeproc{ref-kabacoff2022r}{Kabacoff, 2022}) ,
(\citeproc{ref-grolemund2014hands}{Grolemund, 2014})

\section{References}\label{references}

\phantomsection\label{refs}
\begin{CSLReferences}{1}{0}
\bibitem[\citeproctext]{ref-grolemund2014hands}
Grolemund, G. (2014). \emph{Hands-on programming with r} (p. 45).
O'Reilly Media.

\bibitem[\citeproctext]{ref-kabacoff2022r}
Kabacoff, R. I. (2022). \emph{R in action} (3rd ed., p. 39). Manning
Publications.

\bibitem[\citeproctext]{ref-r_core_team2024}
R Core Team. (2024). \emph{R language definition: Control flow}. R
Foundation for Statistical Computing.
\url{https://cran.r-project.org/doc/manuals/r-release/R-lang.html\#Control-flow}

\end{CSLReferences}

\section{Affidavit}\label{affidavit}

I hereby affirm that this submitted paper was authored unaided and
solely by me. Additionally, no other sources than those in the reference
list were used. Parts of this paper, including tables and figures, that
have been taken either verbatim or analogously from other works have in
each case been properly cited with regard to their origin and
authorship. This paper either in parts or in its entirety, be it in the
same or similar form, has not been submitted to any other examination
board and has not been published.

I acknowledge that the university may use plagiarism detection software
to check my thesis. I agree to cooperate with any investigation of
suspected plagiarism and to provide any additional information or
evidence requested by the university.

Checklist:

\begin{itemize}
\tightlist
\item[$\boxtimes$]
  The handout contains 3-5 pages of text.
\item[$\boxtimes$]
  The submission contains the Quarto file of the handout.
\item[$\boxtimes$]
  The submission contains the Quarto file of the presentation.
\item[$\boxtimes$]
  The submission contains the HTML file of the handout.
\item[$\boxtimes$]
  The submission contains the HTML file of the presentation.
\item[$\boxtimes$]
  The submission contains the PDF file of the handout.
\item[$\boxtimes$]
  The submission contains the PDF file of the presentation.
\item[$\boxtimes$]
  The title page of the presentation and the handout contain personal
  details (name, email, matriculation number).
\item[$\boxtimes$]
  The handout contains a abstract.
\item[$\boxtimes$]
  The presentation and the handout contain a bibliography, created using
  BibTeX with APA citation style.
\item[$\boxtimes$]
  Either the handout or the presentation contains R code that proof the
  expertise in coding.
\item[$\boxtimes$]
  The handout includes an introduction to guide the reader and a
  conclusion summarizing the work and discussing potential further
  investigations and readings, respectively.
\item[$\boxtimes$]
  All significant resources used in the report and R code development.
\item[$\boxtimes$]
  The filled out Affidavit.
\item[$\boxtimes$]
  A concise description of the successful use of Git and GitHub, as
  detailed here: \url{https://github.com/hubchev/make_a_pull_request}.
\item[$\boxtimes$]
  The link to the presentation and the handout published on GitHub.
\end{itemize}

{[}Hoda Haeri,{]} {[}06/18/2025,{]} {[}Koln{]}






\end{document}
